\documentclass[12pt]{article}
\usepackage{fullpage}
\usepackage{amsmath}
\begin{document}
A process $p_i$ waiting to be waken up $c_i$ count later is decoded by a
pair $(p_i,c_i)$. A number of such processes can be stored in a list
$$
(p_1, c_1),
(p_2, c_2),
...
(p_n, c_n).$$
The list is maintained in increasing order of $c_i$, i.e. shorter time
to expire will be found in the beginning of the list.

\section*{List Updating}
When timer of $p_1$ expires, that is when $c_1$ reaches zero the list
will be updated to:
$$
(p_1, 0),
(p_2, c_2 - c_1),
(p_3, c_3 - c_1),
...
(p_n, c_n - c_1).
$$
All other counter $c_i$ has to be substracted by $c_1$ because at this
time the timer has elapsed $c_1$ counts.

Later when timer of $p_2$ expires, the list will be updated to:
$$
(p_1, 0),
(p_2, 0),
(p_3, c_3 - c_1 - c_2 + c_1),
(p_4, c_4 - c_1 - c_2 + c_1),
...
(p_n, c_n - c_1 - c_2 + c_1).
$$
or similarly,
$$
(p_1, 0),
(p_2, 0),
(p_3, c_3 - c_2),
(p_4, c_4 - c_2),
...
(p_n, c_n - c_2).
$$

Again, when timer of $p_3$ expires, the list will be updated to:
$$
(p_1, 0),
(p_2, 0),
(p_3, 0),
(p_4, c_4 - c_3),
(p_5, c_5 - c_3),
...
(p_n, c_n - c_3).
$$

In general, after timer of $p_k$ expires, the list will be updated to:
$$
(p_1, 0),..., (p_k,0),
(p_{k+1}, c_{k+1} - c_k),
...,
(p_n, c_n - c_k).
$$

\section*{Avoiding Global Update}
The patterns show that after $c_k$ expires, timer of process $p_{k+1}$
will always be updated to $c_{k+1} - c_k$. So, to avoid global update
each time a timer expires, the list can be stored initially as:
$$
(p_1, c_1),
(p_2, c_2 - c_1),
(p_3, c_3 - c_2),
...,
(p_n, c_n - c_{n-1}).
$$
Note that this new list is \textit{not} ordered increasingly by the
count element.

\section*{Insertion of a new element}
Suppose a new element $(p_a, c_a)$ is to be inserted to the existing
list
$$
(p_1, c_1),
(p_2, c_2 - c_1),
...,
(p_n, c_n - c_{n-1}).
$$

Depending on the value of $c_a$,
there are three possible cases of insertion to consider:
\begin{enumerate}
\item Insertion at the head of the list, $c_a \le c_1$:

The list should be updated to
$$
(p_a, c_a),
(p_1, c_1 - c_a),
(p_2, c_2 - c_1),
...,
(p_n, c_n - c_{n-1}).
$$
\item Insertion in the middle of the list, $c_k \le c_a < c_{k+1}$:

The list should be updated to
$$
(p_1, c_1),
...,
(p_k, c_k - c_{k-1}),
(p_a, c_a - c_k),
(p_{k+1}, c_{k+1} - c_a),
...,
(p_n, c_n - c_{n-1}).
$$
\item Insertion at the end of the list, $c_n \le c_a$:

The list should be updated to
$$
(p_1, c_1),
(p_2, c_2 - c_1),
...,
(p_n, c_n - c_{n-1}),
(p_a, c_a - c_n).
$$
\end{enumerate}
As can be seen, in all the three cases at most two updates are required.
One for the new element and possibly to the next element right after it.

\section*{How to determine insertion point?}
In the three cases mentioned above, the values of $c_k$, $c_{k+1}$, and
$c_n$ are assumed to be readily accessible, but in fact they are not.
Only the delta of the expiration count are stored in the list.
However, the following observation will apply:

\begin{eqnarray*}
c_k \le& c_a       & < c_{k+1}\\
  0 \le& c_a - c_k & < c_{k+1} - c_k\\
\end{eqnarray*}
Since the value of $c_{k+1} - c_k$ is stored in the list, the search can
stop when this condition is met. But how to compute $c_a - c_k$ since
only $c_k - c_{k-1}$ is available.

During the traversal, when $c_a$ is cumulatively substracted by the
expiration count element in the list, the following values will be
obtained: 
\begin{itemize}
\item After substraction with the first element: $c_a - c_1$.
\item After substraction of the last value with the second element:
$c_a - c_1 - (c_2 - c_1) = c_a - c_2$
\item After continuing the substraction up to the $k$-th element, 
$c_a - c_k$ will be obtained.
\end{itemize}
\end{document}
